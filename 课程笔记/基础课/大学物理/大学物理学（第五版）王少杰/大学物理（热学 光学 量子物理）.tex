\documentclass[12pt,a4paper]{article}

% ======================== 基础宏包 ========================
\usepackage{ctex}          % 中文支持
\usepackage{amsmath}       % 数学公式
\usepackage{amssymb}       % 数学符号
\usepackage{amsfonts}      % 数学字体
\usepackage{geometry}      % 页面设置
\usepackage{enumitem}      % 列表格式
\usepackage{booktabs}      % 专业表格
\usepackage{multirow}      % 表格跨行
\usepackage{xcolor}        % 扩展颜色
\usepackage{ragged2e}      % 文本对齐
\usepackage{bm}            % 加粗向量
\usepackage{upgreek}       % 希腊字母正体
\usepackage{titlesec}      % 标题格式(专业方案)
\usepackage{titletoc}      % 目录格式(仅用这一个包,避免冲突)
\usepackage[bookmarks=true,colorlinks=true]{hyperref}  % 超链接/目录

% ======================== 全局配置 ========================
% 页面设置
\geometry{
    a4paper,
    left=2.5cm,
    right=2.5cm,
    top=2.5cm,
    bottom=2.5cm
}

% 超链接样式
\hypersetup{
    linkcolor=blue,
    filecolor=blue,
    urlcolor=blue,
    citecolor=blue,
    pdfborder={0 0 0},
    bookmarksnumbered=true,
    bookmarksopen=false
}

% 公式编号(章节级)
\numberwithin{equation}{section}

% 列表格式优化
\setlist{
    itemsep=0.2ex,
    parsep=0.5ex,
    leftmargin=2em,
    labelsep=0.8em
}

% ======================== 标题格式(专业版) ========================
% 使用 titlesec 包定义标题
\titleformat{\section}
  {\Large\bfseries\color{blue}}
  {\thesection.}
  {1em}
  {}

\titleformat{\subsection}
  {\large\bfseries\color{cyan}}
  {\thesubsection.}
  {1em}
  {}

\titleformat{\subsubsection}
  {\bfseries\color{teal}}
  {\thesubsubsection.}
  {1em}
  {}

% 调整标题间距
\titlespacing*{\section}{0pt}{1.5ex plus 1ex minus .2ex}{1ex plus .2ex}
\titlespacing*{\subsection}{0pt}{1ex plus .5ex minus .2ex}{0.8ex plus .2ex}
\titlespacing*{\subsubsection}{0pt}{0.8ex plus .3ex minus .2ex}{0.5ex plus .2ex}

% ======================== 目录格式(纯 titletoc 实现,无冲突) ========================
% 1. 美化目录标题(降低字体大小,解决Overfull \vbox)
\renewcommand{\contentsname}{%
    \Large\bfseries\color{blue!80!black} 目录  % 从Huge改为Large,避免垂直溢出
}

% 2. 调整目录整体间距(用titletoc的\vspace实现)
\makeatletter
\renewcommand{\tableofcontents}{%
    \section*{\contentsname}%
    \vspace{-1em}  % 替代\cftbeforetoctitleskip=-1em
    \@starttoc{toc}%
    \vspace{2em}   % 替代\cftaftertoctitleskip=2em
}
\makeatother

% 3. 精细化设置各级目录样式(纯titletoc)
\titlecontents{section}
  [0em]                                  % 整体缩进
  {\large\bfseries\vspace{0.5ex}}        % 字体样式 + 条目上方间距
  {\thecontentslabel\quad}               % 编号样式
  {}                                     % 无编号时的样式
  {\titlerule*[0.5pc]{.}\contentspage}   % 引导线 + 页码

\titlecontents{subsection}
  [2.2em]                                % 缩进(匹配一级编号宽度)
  {\vspace{0.3ex}}                       % 条目上方间距
  {\thecontentslabel\quad}               % 编号样式
  {}
  {\titlerule*[0.5pc]{.}\contentspage}

\titlecontents{subsubsection}
  [4.4em]                                % 缩进(匹配二级编号宽度)
  {\vspace{0.2ex}\small}                 % 间距 + 字体稍小(替代\cftsubsubsecafterpnumskip)
  {\thecontentslabel\quad}               % 编号样式
  {}
  {\titlerule*[0.5pc]{.}\contentspage}

% ======================== 文档内容 ========================
\title{\Huge\bfseries 大学物理核心公式与考点汇总}
\author{浮萍}  % 新增作者:浮萍
\date{}

\begin{document}
\maketitle
\tableofcontents
\clearpage

% ======================== 第一章:热学 ========================
\section{热学}
\subsection{理想气体基本公式}
\begin{itemize}
    \item 理想气体物态方程:$pV=\frac{m}{M}RT$
    \item 压强的微观表达式:$p = \frac{2}{3} n \left( \frac{1}{2} m_0 \overline{v^2} \right) = \frac{2}{3} n \overline{\varepsilon}_{\text{kt}}$($n$为单位体积分子数)
    \item 物态方程另一形式:$p = nkT$($k=\frac{R}{N_A}$为玻尔兹曼常量)
\end{itemize}

\subsection{速率分布函数}
\begin{itemize}
    \item 速率分布函数定义:$\frac{\mathrm{d}N}{N_0} = f(v)\mathrm{d}v$ 或 $f(v) = \frac{\mathrm{d}N}{N_0 \mathrm{d}v}$
    \item 麦克斯韦速率分布函数:
        $$f(v) = 4\pi \left( \frac{m_0}{2\pi kT} \right)^{\frac{3}{2}} e^{-\frac{m_0 v^2}{2kT}} v^2$$
    \item 归一化条件:$\int_{0}^{\infty} f(v) \mathrm{d}v = 1$
\end{itemize}

\subsection{三种统计速率}
\begin{itemize}
    \item 最概然速率:$v_{\text{p}} = \sqrt{\frac{2kT}{m_0}} = \sqrt{\frac{2RT}{M}}$
    \item 平均速率:$\bar{v} = \int_{0}^{\infty} v f(v) \mathrm{d}v = \sqrt{\frac{8kT}{\pi m_0}} = \sqrt{\frac{8RT}{\pi M}}$
    \item 方均根速率:$\sqrt{\overline{v^2}} = \sqrt{\frac{3kT}{m_0}} = \sqrt{\frac{3RT}{M}}$(其中$\overline{v^2} = \int_{0}^{\infty} v^2 f(v) \mathrm{d}v = \frac{3kT}{m_0}$)
\end{itemize}

\subsection{分子动能与自由度}
\begin{itemize}
    \item 平均平动动能:$\overline{\varepsilon}_{\text{kt}} = \frac{1}{2} m_0 \overline{v^2}=\frac{3}{2}kT$
    \item 刚性分子自由度:
        \begin{table}[htbp]
            \centering
            \renewcommand{\arraystretch}{1.2}
            \begin{tabular}{@{}c|c|c|c@{}}
                \toprule
                分子种类 & 平动自由度 & 转动自由度 & 总自由度$i$ \\
                \midrule
                单原子分子 & 3 & 0 & 3 \\
                刚性双原子分子 & 3 & 2 & 5 \\
                刚性多原子分子 & 3 & 3 & 6 \\
                \bottomrule
            \end{tabular}
        \end{table}
    \item 分子平均动能:$\overline{\varepsilon}_{\text{k}} = \frac{i}{2} kT$($i$为总自由度)
    \item 理想气体内能:$E=\frac{m}{M}\frac{i}{2}RT$
\end{itemize}

\subsection{平均自由程与碰撞频率}
\begin{itemize}
    \item 平均自由程:$\overline{\lambda} = \frac{\bar{v}}{\bar{z}}$
    \item 碰撞截面:$\sigma = \pi d^2$($d$为分子直径)
    \item 平均碰撞频率:$\bar{z} = \sqrt{2} \sigma \bar{v} n = \sqrt{2} \pi d^2 \bar{v} n$($\bar{u}=\sqrt{2}\bar{v}$为平均相对速率)
    \item 平均自由程(常用):$\overline{\lambda} = \frac{1}{\sqrt{2} \sigma n} = \frac{1}{\sqrt{2} \pi d^2 n}=\frac{kT}{\sqrt{2}\pi d^2p}$
\end{itemize}

\subsection{热力学过程}
\subsubsection{热容相关}
\begin{itemize}
    \item 摩尔定容热容:$C_{V,\text{m}} = \left( \frac{\text{d}Q}{\text{d}T} \right)_V = \frac{\text{d}E}{\text{d}T} = \frac{i}{2} R$
    \item 摩尔定压热容:$C_{p,\text{m}} = \left( \frac{\text{d}Q}{\text{d}T} \right)_p = \frac{\text{d}E}{\text{d}T} + R$
    \item 迈耶公式:$C_{p,\text{m}} = C_{V,\text{m}} + R$
    \item 比热容比:$\gamma = \frac{C_{p,\text{m}}}{C_{V,\text{m}}}$
\end{itemize}

\subsubsection{等值过程}
\begin{itemize}
    \item 等体过程:
        $$Q_V = \Delta E = \frac{m}{M} C_{V,\text{m}} (T_2 - T_1)$$
    \item 等压过程:
        $$W = p(V_2 - V_1), \quad \Delta E = \frac{m}{M} C_{V,\text{m}} (T_2 - T_1)$$
    \item 等温过程:
        $$W = Q_T = \frac{m}{M}RT\ln\frac{V_2}{V_1} = \frac{m}{M}RT\ln\frac{p_1}{p_2}$$
    \item 绝热过程(泊松公式):
        $$TV^{\gamma-1} = C_2, \quad \frac{p^{\gamma-1}}{T^\gamma} = C_3, \quad pV^\gamma = C_1$$
        $$W = \frac{1}{\gamma-1}(p_1 V_1 - p_2 V_2)$$
        绝热线斜率:$\left( \frac{\mathrm{d}p}{\mathrm{d}V} \right)_Q = -\gamma \frac{p}{V}$,等温线斜率:$\left( \frac{\mathrm{d}p}{\mathrm{d}V} \right)_T = -\frac{p}{V}$
\end{itemize}

\subsection{热机与熵}
\begin{itemize}
    \item 热机效率:$\eta = \frac{W}{Q_1} = 1 - \frac{Q_2}{Q_1}$
    \item 制冷系数:$\varepsilon = \frac{Q_2}{W} = \frac{Q_2}{Q_1 - Q_2}$
    \item 熵的统计意义:$S = k\ln \Omega$
    \item 熵变(热力学):$\mathrm{d}S = \frac{\mathrm{d}Q}{T}$(可逆过程)
    \item 熵差计算:
        $$\Delta S = \frac{m}{M} C_{V,\text{m}} \ln\frac{T_2}{T_1} + \frac{m}{M} R \ln\frac{V_2}{V_1}$$
        $$\Delta S = \frac{m}{M} C_{p,\text{m}} \ln\frac{T_2}{T_1} - \frac{m}{M} R \ln\frac{p_2}{p_1}$$
        可逆绝热过程:$\Delta S = 0$
\end{itemize}

\subsubsection{例题参考}
3、5、6、13、18、20(热学基础);4、5、7、8、9、10、13、15、17、19、21(热力学过程)——《大学物理学(第五版)》王少杰

% ======================== 第二章:光学 ========================
\section{光学}
\subsection{半波损失}
\begin{itemize}
    \item 核心定义:光从光疏介质射向光密介质的反射光,相位突变$\pi$(光程突变$\lambda/2$)。
    \item 产生条件:仅反射光、光疏→光密介质、入射角非零。
    \item 干涉影响:单束半波损失会反转干涉加强/减弱条件。
    \item 应用场景:薄膜干涉、劈尖干涉、牛顿环。
\end{itemize}

\subsection{光的干涉}
\subsubsection{杨氏双缝干涉}
\begin{itemize}
    \item 光程差:$\delta = r_2 - r_1 \approx \frac{xd}{D}$($d$为缝间距,$D$为缝到屏距离)
    \item 明纹条件:$\frac{xd}{D} = \pm k\lambda$ → $x = \pm k \frac{D\lambda}{d}$($k=0,1,2,\dots$)
    \item 暗纹条件:$\frac{xd}{D} = \pm (2k-1)\frac{\lambda}{2}$ → $x = \pm (2k-1)\frac{D\lambda}{2d}$($k=1,2,3,\dots$)
    \item 条纹间距:$\Delta x = \frac{D\lambda}{d}$
\end{itemize}

\subsubsection{薄膜干涉}
\paragraph{等倾干涉}
光程差:$\delta = 2e \sqrt{n_2^2 - n_1^2 \sin^2 i} + \frac{\lambda}{2}$($n_1 < n_2$,半波损失修正)
\begin{itemize}
    \item 明纹:$2e \sqrt{n_2^2 - n_1^2 \sin^2 i} = (k - \frac{1}{2})\lambda$($k=1,2,3,\dots$)
    \item 暗纹:$2e \sqrt{n_2^2 - n_1^2 \sin^2 i} = k\lambda$($k=0,1,2,\dots$)
\end{itemize}

\paragraph{等厚干涉(劈尖)}
空气劈尖光程差:$\delta = 2e + \frac{\lambda}{2}$
\begin{itemize}
    \item 明纹:$2e = (2k-1)\frac{\lambda}{2}$($k=1,2,3,\dots$)
    \item 暗纹:$2e = k\lambda$($k=0,1,2,\dots$),棱边为零级暗纹
    \item 条纹间距:$L = \frac{\lambda}{2\theta}$($\theta$为劈尖角)
\end{itemize}

\paragraph{牛顿环}
光程差:$\delta = 2e + \frac{\lambda}{2}$,几何关系:$e \approx \frac{r^2}{2R}$($R$为透镜曲率半径)
\begin{itemize}
    \item 明环半径:$r_k = \sqrt{(k - \frac{1}{2})\lambda R}$($k=1,2,3,\dots$)
    \item 暗环半径:$r_k = \sqrt{k\lambda R}$($k=0,1,2,\dots$)
    \item 曲率半径测量:$R=\frac{d_{k+m}^{2}-d_{k}^{2}}{4 m \lambda}$($d$为暗环直径)
\end{itemize}

\subsection{光的衍射}
\subsubsection{单缝衍射(夫琅禾费)}
\begin{itemize}
    \item 最大光程差:$\Delta = a\sin\theta$($a$为缝宽,$\theta$为衍射角)
    \item 暗纹条件:$a\sin\theta = \pm k\lambda$($k=1,2,3,\dots$)
    \item 明纹条件:$a\sin\theta = \pm (2k+1)\frac{\lambda}{2}$($k=1,2,3,\dots$)
    \item 中央明纹半角宽度:$\theta_0 \approx \frac{\lambda}{a}$,线宽度:$\Delta x_0 = 2f\frac{\lambda}{a}$($f$为透镜焦距)
    \item 相对光强:$\frac{I}{I_0} = \left( \frac{\sin\alpha}{\alpha} \right)^2$($\alpha = \frac{\pi a\sin\theta}{\lambda}$)
\end{itemize}

\subsubsection{光栅衍射}
\begin{itemize}
    \item 光栅常量:$d = a + b$($a$为缝宽,$b$为刻痕宽)
    \item 光栅方程(主极大):$d\sin\varphi = \pm k\lambda$($k=0,1,2,\dots$)
    \item 缺级条件:$k = \pm \frac{d}{a}k'$($k'=1,2,3,\dots$,同时满足单缝暗纹)
\end{itemize}

\subsubsection{光学仪器分辨本领}
\begin{itemize}
    \item 瑞利判据:最小分辨角$\delta\theta = 1.22 \frac{\lambda}{d}$($d$为孔径)
    \item 分辨本领:$R = \frac{1}{\delta\theta} = \frac{d}{1.22\lambda}$
\end{itemize}

\subsection{光的偏振}
\subsubsection{马吕斯定律}
\begin{itemize}
    \item 自然光通过偏振片:$I_1 = \frac{I_0}{2}$
    \item 线偏振光通过检偏器:$I_2 = I_1 \cos^2\alpha$($\alpha$为偏振方向夹角)
\end{itemize}

\subsubsection{布儒斯特定律}
\begin{itemize}
    \item 起偏振角:$\tan i_0 = \frac{n_2}{n_1}$(此时反射光为线偏振光,$i_0 + r = 90^\circ$)
\end{itemize}

\subsubsection{波片}
\begin{itemize}
    \item 相位差:$\Delta\varphi = \frac{2\pi}{\lambda}(n_o - n_e)d$($n_o$、$n_e$为o光/e光折射率)
    \item 四分之一波片:$\Delta\varphi = (2k+1)\frac{\pi}{2}$,光程差$\delta = (2k+1)\frac{\lambda}{4}$
    \item 半波片:$\Delta\varphi = (2k+1)\pi$,光程差$\delta = (2k+1)\frac{\lambda}{2}$,振动面旋转$2\alpha$
\end{itemize}

\subsubsection{例题参考}
干涉:8、9、11、13、16、18、19、20;衍射:23、24、25、26、28;偏振:32、33、36、38、40——《大学物理学(第五版)》王少杰

% ======================== 第三章:量子物理 ========================
\section{量子物理}
\subsection{早期量子论}
\subsubsection{黑体辐射}
\begin{itemize}
    \item 斯特藩-玻尔兹曼定律:$M_0(T) = \sigma_0 T^4$
    \item 维恩位移定律:$\lambda_m T = b$
\end{itemize}

\subsubsection{光电效应}
\begin{itemize}
    \item 光子能量:$\varepsilon = h\nu$
    \item 爱因斯坦方程:$h\nu = \frac{1}{2}mv_m^2 + A_0$($A_0$为逸出功)
    \item 光子动量:$p = \frac{h\nu}{c} = \frac{h}{\lambda}$
\end{itemize}

\subsubsection{康普顿效应}
\begin{itemize}
    \item 波长改变量:$\Delta\lambda = \lambda' - \lambda = \frac{h}{m_0 c}(1-\cos\theta) = \frac{2h}{m_0 c}\sin^2\frac{\theta}{2}$
    \item 康普顿波长:$\lambda_C = \frac{h}{m_0 c}$
\end{itemize}

\subsubsection{玻尔氢原子理论}
\begin{itemize}
    \item 角动量量子化:$L = mvr = n\hbar$($n=1,2,3,\dots$,$\hbar = \frac{h}{2\pi}$)
    \item 轨道半径:$r_n = a_0 n^2$($a_0$为玻尔半径)
    \item 能级:$E_n = -\frac{13.6}{n^2}\text{eV}$($n=1$时$E_1=-13.6\text{eV}$)
\end{itemize}

\subsubsection{德布罗意假设}
\begin{itemize}
    \item 物质波波长:$\lambda = \frac{h}{p} = \frac{h}{mv}$
    \item 非相对论近似:$\lambda = \frac{h}{\sqrt{2m_0 E_k}}$($E_k$为动能)
    \item 电子波长(加速电压$U$):$\lambda = \frac{1.226}{\sqrt{U}} \text{nm}$
\end{itemize}

\subsection{量子力学基础}
\subsubsection{波函数}
\begin{itemize}
    \item 概率密度:$w = |\Psi|^2 = \Psi^* \cdot \Psi$
    \item 归一化条件:$\iiint_{-\infty}^{+\infty} |\Psi|^2 \mathrm{d}x\mathrm{d}y\mathrm{d}z = 1$
    \item 波函数条件:单值、有限、连续
\end{itemize}

\subsubsection{不确定关系}
\begin{itemize}
    \item 坐标-动量:$\Delta x \Delta p_x \ge \frac{\hbar}{2}$
    \item 时间-能量:$\Delta E \cdot \Delta t \ge \frac{\hbar}{2}$
\end{itemize}

\subsubsection{一维无限深方势阱}
\begin{itemize}
    \item 波函数:$\varPsi_n = \sqrt{\frac{2}{a}}\sin{\frac{n\pi x}{a}}$($a$为势阱宽度)
    \item 能量本征值:$E_n = \frac{n^2 h^2}{8ma^2} = \frac{n^2 \pi^2 \hbar^2}{2ma^2}$($n=1,2,3,\dots$)
\end{itemize}

\subsubsection{线性谐振子}
\begin{itemize}
    \item 能量本征值:$E_n = (n + \frac{1}{2})\hbar\omega$($n=0,1,2,\dots$)
    \item 基态能量:$E_0 = \frac{1}{2}\hbar\omega$
\end{itemize}

\subsubsection{氢原子量子数}
\begin{itemize}
    \item 主量子数:$n=1,2,3,\dots$(决定能量)
    \item 角量子数:$l=0,1,\dots,n-1$,角动量$L = \sqrt{l(l+1)}\hbar$
    \item 磁量子数:$m_l=0,\pm1,\dots,\pm l$,$L_z = m_l\hbar$
    \item 自旋量子数:$s=\frac{1}{2}$,自旋磁量子数$m_s=\pm\frac{1}{2}$,$S_z = \pm\frac{1}{2}\hbar$
\end{itemize}

\subsubsection{例题参考}
早期量子论:2、4、7、8、12、14、17、18;量子力学:19、20、21、22、26、27——《大学物理学(第五版)》王少杰

\end{document}